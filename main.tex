\documentclass[lettersize,journal]{IEEEtran}
\usepackage{amsmath,amsfonts}
\usepackage{algorithmic}
\usepackage{algorithm}
\usepackage{array}
\usepackage[caption=false,font=normalsize,labelfont=sf,textfont=sf]{subfig}
\usepackage{textcomp}
\usepackage{stfloats}
\usepackage{url}
\usepackage{verbatim}
\usepackage{graphicx}
\usepackage{cite}
\hyphenation{op-tical net-works semi-conduc-tor IEEE-Xplore}

\begin{document}

\title{OUXT Polaris: Autonomous Navigation System for the 2022 Maritime RobotX Challenge}
\author{
    Kenta Okamoto,Kyoto Institute of Tech., m2623106@edu.kit.ac.jp \\ \and
    Akihisa Nagata, Kansai Univ. , k065604kansai@gmail.com \\ \and
    Masato Kobayashi, Kobe Univ. , 171w951w@gsuite.kobe-u.ac.jp \\ \and
    Shunya Tanaka, , syun111@gmail.com \\ \and
    Masaya Kataoka, TEIR IV inc . , ms.kataoka@gmail.com,
}

% The paper headers
\markboth{IEEE TRANSACTIONS ON XXXXX,~Vol.~xx, No.~x, XXX~2022}%
{Shell \MakeLowercase{\textit{et al.}}: A Sample Article Using IEEEtran.cls for IEEE Journals}

% \IEEEpubid{0000-0000/00\$00.00~\copyright~2021 IEEE}
% Remember, if you use this you must call \IEEEpubidadjcol in the second
% column for its text to clear the IEEEpubid mark.

\maketitle

\begin{abstract}
OUXT-Polaris has been developing an autonomous navigation system by participating in the 
Maritime RobotX Challenge 2014, 2016, and 2018. 
In this paper, we describe the improvement of the previous vessel system. 
We also indicate the advantage of the improved design.
Moreover, we describe the developing method for Covid-19 and the 
feature components for the next RobotX Challenge.
\end{abstract}

\begin{IEEEkeywords}
Maritime systems, Robotics, Unmanned surface vehicle
\end{IEEEkeywords}

\section{Introduction}
First of all, we are motivated to develop a big field robot in a large area such as the ocean.
In recent years, the aging and shrinking population, as well as a shortage of workers,
has led to an increase in demand for the automation of cars, robots, and other equipment.
Among these, automated driving is being developed with particular emphasis.
Moving the autonomous vehicle or robot outside has a very severe problem.
They need to hedge unknown obstacles and go to the target position.
The environment such as weather, temperature, or underwater around robots causes sensor and hardware problems.
There are each challenging problems and They are also interesting for us.  
In this competition, we have a chance to develop a system to get over the wild environment 
for the robots on the ocean. Therefore, we are participating in the Maritime RobotX Challenge.

\section{Conclusion}
XXXXXX
% \section*{Acknowledgments}
\begin{thebibliography}{1}
\bibliographystyle{IEEEtran}

  \bibitem{YOLOX}
  Ge, Zheng, et al. "Yolox: Exceeding yolo series in 2021." arXiv preprint arXiv:2107.08430 (2021).

  \bibitem{RobotX2018_video}
  \url{https://www.youtube.com/watch?v=MqDBxzS4uy4}

\end{thebibliography}

\vfill

\end{document}