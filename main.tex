\documentclass[lettersize,journal]{IEEEtran}
\usepackage{amsmath,amsfonts}
\usepackage{algorithmic}
\usepackage{algorithm}
\usepackage{array}
\usepackage[caption=false,font=normalsize,labelfont=sf,textfont=sf]{subfig}
\usepackage{textcomp}
\usepackage{stfloats}
\usepackage{url}
\usepackage{verbatim}
\usepackage{graphicx}
\usepackage{cite}
\hyphenation{op-tical net-works semi-conduc-tor IEEE-Xplore}

\begin{document}

\title{OUXT Polaris: Autonomous Navigation System for the 2022 Maritime RobotX Challenge}
\author{
    Kenta Okamoto,Kyoto Institute of Tech., m2623106@edu.kit.ac.jp \\ \and
    Akihisa Nagata, Kansai Univ. , k065604@kansai-u.ac.jp \\ \and
    Masato Kobayashi, Kobe Univ. , 171w951w@gsuite.kobe-u.ac.jp \\ \and
    Yusei Nagao, Osaka Institute of Tech., m1m22r23@oit.ac.jp \\ \ and
    Shunya Tanaka, , syun111@gmail.com \\ \and
    Masaya Kataoka, TEIR IV inc . , ms.kataoka@gmail.com, \\ \and
    Kyoma Arai, Tokai Univ. , 2cemm007@mail.u-tokai.ac.jp,
}

% The paper headers
\markboth{IEEE TRANSACTIONS ON XXXXX,~Vol.~xx, No.~x, XXX~2022}%
{Shell \MakeLowercase{\textit{et al.}}: A Sample Article Using IEEEtran.cls for IEEE Journals}

% \IEEEpubid{0000-0000/00\$00.00~\copyright~2021 IEEE}
% Remember, if you use this you must call \IEEEpubidadjcol in the second
% column for its text to clear the IEEEpubid mark.

\maketitle

\begin{abstract}
OUXT-Polaris has been developing an autonomous navigation system by participating in the 
Maritime RobotX Challenge 2014, 2016, and 2018. 
In this paper, we describe the improvement of the previous vessel system. 
We also indicate the advantage of the improved design.
Moreover, we describe the developing method for Covid-19 and the 
feature components for the next RobotX Challenge.
\end{abstract}

\begin{IEEEkeywords}
Maritime systems, Robotics, Unmanned surface vehicle
\end{IEEEkeywords}

\section{Introduction}
First of all, we are motivated to develop a big field robot in a large area such as the ocean.
In recent years, the aging and shrinking population, as well as a shortage of workers,
has led to an increase in demand for the automation of cars, robots, and other equipment.
Among these, automated driving is being developed with particular emphasis.
Moving the autonomous vehicle or robot outside has a very severe problem.
They need to hedge unknown obstacles and go to the target position.
The environment such as weather, temperature, or underwater around robots causes sensor and hardware problems.
There are each challenging problems and They are also interesting for us.  
In this competition, we have a chance to develop a system to get over the wild environment 
for the robots on the ocean. Therefore, we are participating in the Maritime RobotX Challenge.

\section{Over All Improved System}
\section{Hardware Developments}
\section{Software Developments}
\subsection{ROS2-based Autonomous Navigation Stack}
In “Maritime RobotX Challenge 2018”, we used Robot Operating System 1(ROS1) for developing software.
However, the development of ROS1 was finished with python2 end of life.
Therefore, we adopted the next generation of ROS called “ROS2”.
As shown in Fig. 1, our ROS2-based simulation and software system was already developed.
Our software contributions are listed as follows.

\begin{itemize}
  \item {\it Software System }:
    We rebuilt the software system from ROS1 to ROS2
  \item {\it Behavior Tree}:
    We adopted the behavior tree library in ROS2 and built our original behavior tree.
  \item {\it Simulation Tool Development }:
    We developed LiDAR simulation by using intel ray-tracing OSS "Embree".
  \item {\it CI/CD}:
    CI/CD systems are developed on the GitHub Actions Platform.
  \item {\it Object Detection}:
    We are developing a lidar-camera fusion object detection system for this project.
\end{itemize}
We published all codes in GitHub to give feedback knowledge to the ROS community and 
Open-Source all our resources not only software but also including CAD models, and circuit data.

\subsection{Software System Architecture}
Our navigation stack is based on ROS2, but we do not use the navigation2 library. We develop our original software.
Our software is highly modularized, so some of our members use our stacks in other autonomous mobility competitions.

\subsection{Object Detection}
We used  YOLOX\cite{YOLOX} for object Detection of task object information such as buoys and docks.
We created annotation data based on images and videos obtained from past RobotX Challenges.
The result of training and inference is shown in Fig. hoge. 
The image is part of a video recorded during our challenging navigation in the 2018 RobotX Challenge.\cite{{RobotX2018_video}}

\section{Conclusion}
XXXXXX
% \section*{Acknowledgments}
\begin{thebibliography}{1}
\bibliographystyle{IEEEtran}

  \bibitem{YOLOX}
  Ge, Zheng, et al. "Yolox: Exceeding yolo series in 2021." arXiv preprint arXiv:2107.08430 (2021).

  \bibitem{RobotX2018_video}
  \url{https://www.youtube.com/watch?v=MqDBxzS4uy4}

\end{thebibliography}

\vfill

\end{document}
